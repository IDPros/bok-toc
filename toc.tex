\hypertarget{introduction}{%
\chapter{Introduction}\label{introduction}}

\hypertarget{introduction-to-identity-part-1-admin-time-article-in-progress}{%
\section{\texorpdfstring{Introduction to Identity -- Part 1:
Admin-time \emph{-- article in
progress}}{Introduction to Identity -- Part 1: Admin-time -- article in progress}}\label{introduction-to-identity-part-1-admin-time-article-in-progress}}

Abstract: This article introduces the concepts of digital identity and
identity and access management (IAM). It also discusses the constituents
that identity professionals serve, compares and contrasts
business-to-employee (B2E) and business-to-consumer (B2C) identity use
cases, and considers IAM technologies from the perspective of
administrative, or admin-time, technologies.

Sections in this article include:

\begin{itemize}
\item
  Introduction: How to Approach Identity and IAM.
\item
  Constituencies - who is it that we serve?
\item
  Business-to-Employee (B2E): Making Employees Productive.
\item
  Business-to-Business (B2B): Connecting to Partners.
\item
  Business-to-Consumer (B2C): Digitally Engage.
\item
  Technologies Involved - Admin-time vs. Run-time.
\item
  Admin-time Technologies.
\item
  Sources of ``Truth''.
\item
  Identity Governance and Administration.
\item
  Identity Analytics.
\item
  Privileged Account Management.
\item
  Identity Proofing.
\end{itemize}

\hypertarget{introduction-to-identity-part-2-run-time-article-in-progress}{%
\section{\texorpdfstring{Introduction to Identity -- Part 2: Run-time
\emph{-- article in
progress}}{Introduction to Identity -- Part 2: Run-time -- article in progress}}\label{introduction-to-identity-part-2-run-time-article-in-progress}}

\hypertarget{introduction-to-identity-part-3-use-cases-article-in-progress}{%
\section{\texorpdfstring{Introduction to Identity -- Part 3: Use
Cases \emph{-- article in
progress}}{Introduction to Identity -- Part 3: Use Cases -- article in progress}}\label{introduction-to-identity-part-3-use-cases-article-in-progress}}

\hypertarget{ethics-article-in-progress}{%
\section{\texorpdfstring{Ethics \emph{-- article in
progress}}{Ethics -- article in progress}}\label{ethics-article-in-progress}}

\hypertarget{information-security}{%
\section{Information Security}\label{information-security}}

\hypertarget{trust-in-the-iam-context}{%
\section{Trust in the IAM Context}\label{trust-in-the-iam-context}}

\hypertarget{privacy}{%
\section{Privacy}\label{privacy}}

\hypertarget{identification-and-authentication}{%
\section{Identification and
authentication}\label{identification-and-authentication}}

\hypertarget{context-and-identity}{%
\subsection{Context and Identity}\label{context-and-identity}}

\hypertarget{levels-of-assurance}{%
\subsection{Levels of Assurance}\label{levels-of-assurance}}

\hypertarget{digital-identity}{%
\chapter{Digital Identity}\label{digital-identity}}

\hypertarget{definition-of-digital-identity}{%
\section{Definition of Digital
Identity}\label{definition-of-digital-identity}}

Abstract: Despite the difficulty of creating a universal definition of
identity, we create a working definition of a more limited concept of
digital identity.~ In this section, we focus on human persons and touch
only slightly on non-personal identities such as corporations and
devices. Starting with the concept that digital identity is a unique
identifier together with relevant attributes required to enable the
identifier to be used in the context of a digital transaction, this
article elaborates and articulates interesting details, such as the
level of certainty about and provenance of attribute values.

\hypertarget{reputation}{%
\subsection{Reputation}\label{reputation}}

\hypertarget{digital-identifiers-article-in-progress}{%
\section{\texorpdfstring{Digital Identifiers \emph{-- article in
progress}}{Digital Identifiers -- article in progress}}\label{digital-identifiers-article-in-progress}}

Abstract: What is in a name?~ It turns out that there are concerns that
are explored here. These include the domain in which it can be
considered unique when it can be reused, whether it should be considered
secret, and whether it should be memorable.~ Additional system-level
considerations are raised such as permanent system identifiers. Given
that users may forget or lose their identifiers, the article also
discusses the need to allow for the safe recovery of the same.
Identifiers for devices are covered more fully in the non-human entity
section.

\hypertarget{digital-identity-lifecycle}{%
\section{Digital Identity
Lifecycle}\label{digital-identity-lifecycle}}

Abstract: In addition to the steps typically associated with other
digital records, such as create, update and delete, this article
describes several other activities also associated with digital
identities.~ For instance, there are activities that may gather or
dispose of additional attribute information either based on claims made
by a person or based on information from 3rd parties. This article
provides a list of activities that may occur between the creation of the
digital identity and its disposal.

\hypertarget{proofing-binding-or-registration}{%
\section{Proofing, Binding or Registration
}\label{proofing-binding-or-registration}}

Abstract: In many contexts, it is important to relate a human to a
digital account.~ Typically it matters in commercial and institutional
environments. This activity has been described as proofing or vetting,
implying certainty about the mapping. But there is a gradient of need -
in some cases, it is very important such as in the fields of medicine or
finance, whereas in other cases much less care is needed to achieve the
needed level of assurance. This article discusses the drivers and the
palette of tactics that can be used to balance the desired level of
certainty to the mapping and the desired level of friction to be
experienced by the user.

\hypertarget{verificationvalidation}{%
\subsection{Verification/Validation}\label{verificationvalidation}}

\hypertarget{credentials}{%
\section{Credentials}\label{credentials}}

Abstract: When the registration process contains more than a little
friction, many systems provide a way to avoid that friction during
logins, a process that happens many more times than registration does.
In the simplest scenario, this is done by issuing a user ID and a
password, in other words, a credential. This section describes the
varieties of credentials that are in common use.~ It also describes
methods for establishing credentials (how to convey them safely) and
some recovery mechanisms when they are lost or compromised. Because
credentials can be stolen, this article touches on the approach that
some implementations have taken which look to device identities to
reduce risk.

\hypertarget{access-control}{%
\chapter{Access Control}\label{access-control}}

\hypertarget{introduction-to-access-control-article-in-progress}{%
\section{\texorpdfstring{Introduction to Access Control \emph{--
article in
progress}}{Introduction to Access Control -- article in progress}}\label{introduction-to-access-control-article-in-progress}}

Abstract: As the name implies Identity and Access Management (IAM) is
split into two functions: managing identity information and performing
access control. Arguably, if there was no access control requirement
there would be no need for identity management; it is therefore the
focus for IAM professionals.

At its core access control is ensuring users are authenticated to access
protected resources. This is accomplished by managing user entitlements
and satisfying the requirements of relying applications so that users
can only access the systems and information they are entitled to access.

This article looks at the history of access management, the expected
current functionality and the trends to be expected.

\hypertarget{authentication-1}{%
\subsection{Authentication}\label{authentication-1}}

\hypertarget{dynamic-authentication-risk-based}{%
\subsection{Dynamic Authentication
(risk-based)}\label{dynamic-authentication-risk-based}}

\hypertarget{multi-factor-authentication}{%
\subsection{Multi-Factor
Authentication}\label{multi-factor-authentication}}

\hypertarget{single-sign-on-within-a-domain}{%
\subsection{Single Sign-on Within a
Domain}\label{single-sign-on-within-a-domain}}

\hypertarget{centralised-authentication-service}{%
\subsection{Centralised Authentication
Service}\label{centralised-authentication-service}}

\hypertarget{federated-authentication-between-domains}{%
\subsection{Federated Authentication (between
domains)}\label{federated-authentication-between-domains}}

\hypertarget{device-identity-for-corroboration}{%
\subsection{Device Identity for
Corroboration}\label{device-identity-for-corroboration}}

\hypertarget{fast-identity-online-fido-and-its-cousins}{%
\subsection{Fast Identity Online (FIDO) and its
cousins}\label{fast-identity-online-fido-and-its-cousins}}

\hypertarget{session-management}{%
\subsection{Session Management}\label{session-management}}

\hypertarget{authorization}{%
\section{Authorization}\label{authorization}}

\hypertarget{resources-to-protect}{%
\subsection{Resources to Protect}\label{resources-to-protect}}

\hypertarget{authorisation}{%
\subsection{Authorisation}\label{authorisation}}

\hypertarget{acls}{%
\subsubsection{ACL's}\label{acls}}

\hypertarget{rbac}{%
\subsubsection{RBAC}\label{rbac}}

\hypertarget{abac-dynamic-access-management}{%
\subsubsection{ABAC / Dynamic Access
Management}\label{abac-dynamic-access-management}}

\hypertarget{policy-management-solutions}{%
\paragraph{Policy Management
solutions}\label{policy-management-solutions}}

\hypertarget{privileged-access-management}{%
\subsection{Privileged Access
Management}\label{privileged-access-management}}

\hypertarget{alignment-to-risk-management}{%
\subsubsection{Alignment to Risk
Management}\label{alignment-to-risk-management}}

\hypertarget{system-accounts}{%
\subsubsection{System Accounts}\label{system-accounts}}

\hypertarget{laws-regulations-and-standards}{%
\chapter{Laws, Regulations, and
Standards}\label{laws-regulations-and-standards}}

Abstract: This chapter provides information about the externally defined
environment in which Identity and Access management professionals
operate.~ The laws are documents that define duties and consequences in
legal jurisdictions, such as countries. Regulations are more specific
and detailed requirements.~ Standards may also be mandatory; government
entities often require compliance with standards produced by certain
standards bodies. We also include \emph{de facto} standards and
recommended practices here.

\hypertarget{framework-to-understand-legal-environment-article-in-progress}{%
\section{\texorpdfstring{Framework to Understand Legal Environment
\emph{-- article in
progress}}{Framework to Understand Legal Environment -- article in progress}}\label{framework-to-understand-legal-environment-article-in-progress}}

Abstract: Identity systems and its participants are governed by a myriad
and complex set of laws, regulations, and contractual requirements, and
the obligations they impose are not always clear. This article focuses
on the legal environment that governs identity systems.~ The emphasis is
on United States, but references are made to other countries' laws and
efforts to coordinate rules underway in the UN Commission on
International Trade Law (UNCITRAL) regarding identity management
legislation.

\hypertarget{approach-to-compliance-for-the-identity-practitioner}{%
\section{Approach to Compliance for the Identity
Practitioner}\label{approach-to-compliance-for-the-identity-practitioner}}

Abstract:

The overview, above, provides a broad perspective on what the
practitioner might encounter. This article provides a companion piece
that is less theoretical and more practical and concise. This does not
provide legal advice; for that one must consult a legal professional.
Instead we chart paths that the reader might take in sample situations
to prepare for legal review. The goal is to ensure the identity system,
as built and operated, will be in robust compliance with law. This takes
the form of three illustrative use-cases where the identity system
supports various combinations of jurisdictions, participants and
federation:

a) Single jurisdiction, supporting customer access, including out-bound
federation for certain aspects of the customer journey;

b) A system that relies entirely on external "identity providers", with
operations in several jurisdictions;

c) A multi-jurisdiction employee/contractor-focused system, which wishes
to use biometric techniques for authentication.

The general approach is to use the jurisdictions, participants,
federations and technologies under consideration in order to locate
aspects of the law that must be considered.

\hypertarget{highlights-of-selected-laws}{%
\section{Highlights of Selected
Laws}\label{highlights-of-selected-laws}}

Abstract: This section is organized by jurisdiction.~ It is intended to
provide at a minimum a reference to known laws and regulations in
jurisdictions likely to be encountered by our membership.~ At present
this includes Europe, United States, and Canada will likely also include
Australia in the short term.

\hypertarget{europe}{%
\subsection{Europe}\label{europe}}

\hypertarget{introduction-to-gdpr-article-in-progress}{%
\subsubsection{\texorpdfstring{Introduction to GDPR \emph{-- article in
progress}}{Introduction to GDPR -- article in progress}}\label{introduction-to-gdpr-article-in-progress}}

Abstract: This article provides a basic understanding of how the
\emph{General Data Protection Regulation (GDPR)} applies when processing
`any information relating to an identified or identifiable natural
person'.

\hypertarget{iam-implications-of-gdpr-article-in-progress}{%
\subsubsection{\texorpdfstring{IAM Implications of GDPR \emph{-- article in
progress}}{IAM Implications of GDPR -- article in progress}}\label{iam-implications-of-gdpr-article-in-progress}}

Abstract: This article provides information to the IAM practitioner
about how to achieve compliance with the European data protection and
privacy rules for European and multi-national firms

\hypertarget{united-states}{%
\subsection{United States}\label{united-states}}

Abstract:~ This article explains how identity and access management
supports the requirements of prominent U.S. laws.

\hypertarget{sarbanes-oxley-section-404}{%
\subsubsection{Sarbanes-Oxley Section
404}\label{sarbanes-oxley-section-404}}

\hypertarget{health-insurance-portability-and-accountability-act-hipaa}{%
\subsubsection{Health Insurance Portability and Accountability Act
(HIPAA)}\label{health-insurance-portability-and-accountability-act-hipaa}}

\hypertarget{health-information-technology-for-economic-and-clinical-health-act-hitech}{%
\subsubsection{Health Information Technology for Economic and Clinical
Health Act
(HITECH)}\label{health-information-technology-for-economic-and-clinical-health-act-hitech}}

\hypertarget{family-educational-rights-and-privacy-act-of-1974-ferpa}{%
\subsubsection{Family Educational Rights and Privacy Act of 1974
(FERPA)}\label{family-educational-rights-and-privacy-act-of-1974-ferpa}}

\hypertarget{childrens-online-privacy-protection-act-coppa}{%
\subsubsection{Children's Online Privacy Protection Act
(COPPA)}\label{childrens-online-privacy-protection-act-coppa}}

\hypertarget{fair-and-accurate-credit-transaction-act-facta}{%
\subsubsection{Fair and Accurate Credit Transaction Act
(FACTA)}\label{fair-and-accurate-credit-transaction-act-facta}}

\hypertarget{canada}{%
\subsection{Canada}\label{canada}}

Abstract:~ This article explains how identity and access management
support the requirements of prominent Canadian laws.

\hypertarget{personal-information-protection-and-electronic-documents-act-piped-act-or-pipeda}{%
\subsubsection{Personal Information Protection and Electronic Documents Act
(PIPED Act, or
PIPEDA)}\label{personal-information-protection-and-electronic-documents-act-piped-act-or-pipeda}}

\hypertarget{regulations}{%
\section{Regulations}\label{regulations}}

Abstract:~ This article explains how identity and access management
supports the requirements of prominent regulations.

\hypertarget{standards}{%
\section{Standards}\label{standards}}

Abstract: There are many standards. Standards may be mandatory such as
when government entities require compliance with standards produced by
certain standards bodies.~ We also include \emph{de facto} standards and
recommended practices here. This is a curated set of standards that have
been deemed to be useful to identity professionals.~ They are organized
topically, not by their source. Standards that span more than one topic
are possible. In this case cross references may be used.

\hypertarget{architecture}{%
\subsection{Architecture}\label{architecture}}

Abstract: This article surveys the known standards concerning
architecture for identity systems.

\hypertarget{isoiec-24760-22015-information-technology----security-techniques----a-framework-for-identity-management----part-2-reference-architecture-and-requirements-article-in-progress}{%
\subsubsection{\texorpdfstring{ISO/IEC 24760-2:2015 Information technology
-\/- Security techniques -\/- A framework for identity management -\/-
Part 2: Reference architecture and requirements -- \emph{article in
progress}}{ISO/IEC 24760-2:2015 Information technology -\/- Security techniques -\/- A framework for identity management -\/- Part 2: Reference architecture and requirements -- article in progress}}\label{isoiec-24760-22015-information-technology----security-techniques----a-framework-for-identity-management----part-2-reference-architecture-and-requirements-article-in-progress}}

\hypertarget{assurance}{%
\subsection{Assurance}\label{assurance}}

Abstract: This article surveys the known standards concerning risk and
assurance for identity systems.

\hypertarget{directive-on-identity-management---appendix-a-standard-on-identity-and-credential-assurance}{%
\subsubsection{\texorpdfstring{\emph{Directive on~Identity Management~-
Appendix A: Standard on Identity and Credential
Assurance}}{Directive on~Identity Management~- Appendix A: Standard on Identity and Credential Assurance}}\label{directive-on-identity-management---appendix-a-standard-on-identity-and-credential-assurance}}

{[}Canada{]} Government of Canada July 2019
\url{https://www.tbs-sct.gc.ca/pol/doc-eng.aspx?id=32612}

\hypertarget{digital-identity-guidelines}{%
\subsubsection{\texorpdfstring{\emph{Digital Identity
Guidelines}}{Digital Identity Guidelines}}\label{digital-identity-guidelines}}

{[}SP 800-63-3{]}~~~ NIST Special Publication 800-63-3~~~ June 2017~~~
https://doi.org/10.6028/NIST.SP.800-63-3~~~

\hypertarget{guide-for-applying-the-risk-management-framework-to-federal-information-systems-a-security-life-cycle-approach}{%
\subsubsection{\texorpdfstring{\emph{Guide for Applying the Risk Management
Framework to Federal Information Systems: A Security Life Cycle
Approach}}{Guide for Applying the Risk Management Framework to Federal Information Systems: A Security Life Cycle Approach}}\label{guide-for-applying-the-risk-management-framework-to-federal-information-systems-a-security-life-cycle-approach}}

{[}SP-800-37{]}~~~ NIST Special Publication 800-37r1~~~ June 2014~~~
https://doi.org/10.6028/NIST.SP.800-37r1~~~

\hypertarget{authentication-2}{%
\subsection{Authentication}\label{authentication-2}}

Abstract: This article surveys the known standards concerning methods of
authenticating principals.

\hypertarget{digital-identity-guidelines-authentication-and-lifecycle-management}{%
\subsubsection{\texorpdfstring{\emph{Digital Identity Guidelines:
Authentication and Lifecycle
Management}}{Digital Identity Guidelines: Authentication and Lifecycle Management}}\label{digital-identity-guidelines-authentication-and-lifecycle-management}}

{[}SP 800-63B{]}~~~ NIST Special Publication 800-63C~~~ December 2017~~~
https://doi.org/10.6028/NIST.SP.800-63b~~~

\hypertarget{introduction-to-public-key-technology-and-the-federal-pki-infrastructure}{%
\subsubsection{\texorpdfstring{\emph{Introduction to Public Key Technology
and the Federal PKI
Infrastructure}}{Introduction to Public Key Technology and the Federal PKI Infrastructure}}\label{introduction-to-public-key-technology-and-the-federal-pki-infrastructure}}

{[}SP 800-32{]}~~~ NIST Special Publication 800-32~~~ February 2001.~~~
https://tsapps.nist.gov/publication/get\_pdf.cfm?pub\_id=151247~~~

\hypertarget{lightweight-directory-access-protocol-ldap-technical-specification-road-map}{%
\subsubsection{\texorpdfstring{\emph{Lightweight Directory Access Protocol
(LDAP): Technical Specification Road
Map}}{Lightweight Directory Access Protocol (LDAP): Technical Specification Road Map}}\label{lightweight-directory-access-protocol-ldap-technical-specification-road-map}}

{[}IETF RFC 4510{]}~~~ RFC 4510~~~ June 2006~~~
https://tools.ietf.org/html/rfc4510~~~

\hypertarget{openid-connect-core-1.0-incorporating-errata-set-1}{%
\subsubsection{\texorpdfstring{\emph{OpenID Connect Core 1.0 incorporating
errata set
1}}{OpenID Connect Core 1.0 incorporating errata set 1}}\label{openid-connect-core-1.0-incorporating-errata-set-1}}

{[}OIDC{]}~~~ Sakimura, N., Bradley, B., Jones, M., de Medeiros, B., and
C. Mortimore~~~ November 2014~~~
https://openid.net/specs/openid-connect-core-1\_0.html.~~~

\hypertarget{personal-identity-verification-piv-of-federal-employees-and-contractors}{%
\subsubsection{\texorpdfstring{\emph{Personal Identity Verification (PIV) of
Federal Employees and
Contractors}}{Personal Identity Verification (PIV) of Federal Employees and Contractors}}\label{personal-identity-verification-piv-of-federal-employees-and-contractors}}

{[}FIPS 201-2{]}~~~ NIST FIPS Publication 201-2~~~ September 2013~~~
https://doi.org/10.6028/NIST.FIPS.201-2~~~

\hypertarget{biometric-data-specification-for-personal-identity-verification}{%
\subsubsection{\texorpdfstring{\emph{Biometric Data Specification for
Personal Identity
Verification}}{Biometric Data Specification for Personal Identity Verification}}\label{biometric-data-specification-for-personal-identity-verification}}

{[}SP 800-76-2{]}~~~ NIST Special Publication 800-76-2~~~ July 2013~~~
https://doi.org/10.6028/NIST.SP.800-76-2~~~

\hypertarget{authorization-1}{%
\subsection{Authorization}\label{authorization-1}}

Abstract: This article surveys the known standards concerning methods of
access control. These standards involve protecting resources.~ This is
sometimes called authorization.

\hypertarget{the-oauth-2.0-authorization-framework}{%
\subsubsection{\texorpdfstring{\emph{The OAuth 2.0 Authorization
Framework}}{The OAuth 2.0 Authorization Framework}}\label{the-oauth-2.0-authorization-framework}}

{[}IETF RFC 6749{]}~~~ RFC 6749~~~ October 2012~~~
https://tools.ietf.org/html/rfc6749~~~

\hypertarget{user-managed-access-uma-profile-of-oauth-2.0}{%
\subsubsection{\texorpdfstring{\emph{User-Managed Access (UMA) Profile of
OAuth
2.0}}{User-Managed Access (UMA) Profile of OAuth 2.0}}\label{user-managed-access-uma-profile-of-oauth-2.0}}

Abstract: The weaknesses of many notice-and-consent paradigms of data
privacy are clear. This article notes the social, legal and regulatory
drivers and examines some approaches to satisfy them.

{[}KI UMA{]}~~~ Kantara Initiative UMA Recommendation~~~ December
2015~~~ https://docs.kantarainitiative.org/uma/rec-uma-core.html~~~

\hypertarget{federation}{%
\subsection{Federation}\label{federation}}

Abstract: This article surveys the known standards concerning methods of
allowing authentication from one domain to be honored in another.

\hypertarget{openid-connect-core-1.0-incorporating-errata-set-1-1}{%
\subsubsection{\texorpdfstring{\emph{OpenID Connect Core 1.0 incorporating
errata set
1}}{OpenID Connect Core 1.0 incorporating errata set 1}}\label{openid-connect-core-1.0-incorporating-errata-set-1-1}}

{[}OIDC{]}~~~ Sakimura, N., Bradley, B., Jones, M., de Medeiros, B., and
C. Mortimore~~~ November 2014~~~
https://openid.net/specs/openid-connect-core-1\_0.html.~~~

\hypertarget{assertions-and-protocols-for-the-oasis-security-assertion-markup-language-saml-v2.0}{%
\subsubsection{\texorpdfstring{\emph{Assertions and Protocols for the OASIS
Security Assertion Markup Language (SAML)
V2.0}}{Assertions and Protocols for the OASIS Security Assertion Markup Language (SAML) V2.0}}\label{assertions-and-protocols-for-the-oasis-security-assertion-markup-language-saml-v2.0}}

{[}OASIS SAML 2{]}~~~ SAML 2.0~~~ March 2005~~~
http://docs.oasis-open.org/security/saml/v2.0/saml-core-2.0-os.pdf~~~

\hypertarget{digital-identity-guidelines-federation-and-assertions}{%
\subsubsection{\texorpdfstring{\emph{Digital Identity Guidelines: Federation
and
Assertions}}{Digital Identity Guidelines: Federation and Assertions}}\label{digital-identity-guidelines-federation-and-assertions}}

{[}SP 800-63C{]}~~~ NIST Special Publication 800-63C~~~ December 2017~~~
https://doi.org/10.6028/NIST.SP.800-63c~~~

\hypertarget{lifecycle}{%
\subsection{Lifecycle}\label{lifecycle}}

Abstract: This article surveys the known standards concerning the
creation and registration of identities and subsequent changes to the
characteristics of those identities and the eventual removal of the
same.

\hypertarget{standard-on-identity-and-credential-assurance}{%
\subsubsection{\texorpdfstring{\emph{Standard on Identity and Credential
Assurance}}{Standard on Identity and Credential Assurance}}\label{standard-on-identity-and-credential-assurance}}

{[}Canada{]}~~~ Government of Canada~~~ July 2019~~~
https://www.tbs-sct.gc.ca/pol/doc-eng.aspx?id=32612

\hypertarget{digital-identity-guidelines-enrollment-and-identity-proofing-requirements}{%
\subsubsection{\texorpdfstring{\emph{Digital Identity Guidelines: Enrollment
and Identity Proofing
Requirements}}{Digital Identity Guidelines: Enrollment and Identity Proofing Requirements}}\label{digital-identity-guidelines-enrollment-and-identity-proofing-requirements}}

{[}SP 800-63A{]}~~~ NIST Special Publication 800-63A~~~ December 2017~~~
https://doi.org/10.6028/NIST.SP.800-63a~~~

\hypertarget{digital-identity-guidelines-authentication-and-lifecycle-management-1}{%
\subsubsection{\texorpdfstring{\emph{Digital Identity Guidelines:
Authentication and Lifecycle
Management}}{Digital Identity Guidelines: Authentication and Lifecycle Management}}\label{digital-identity-guidelines-authentication-and-lifecycle-management-1}}

{[}SP 800-63B{]}~~~ NIST Special Publication 800-63C~~~ December 2017~~~
https://doi.org/10.6028/NIST.SP.800-63b~~~

\hypertarget{system-for-cross-domain-identity-management-protocol}{%
\subsubsection{\texorpdfstring{\emph{System for Cross-domain Identity
Management:
Protocol}}{System for Cross-domain Identity Management: Protocol}}\label{system-for-cross-domain-identity-management-protocol}}

{[}IETF RFC 7644{]}~~~RFC 7644~~~September
2015~~~https://tools.ietf.org/html/rfc7644

\hypertarget{system-for-cross-domain-identity-management-core-schema}{%
\subsubsection{\texorpdfstring{\emph{System for Cross-domain Identity
Management: Core
Schema}}{System for Cross-domain Identity Management: Core Schema}}\label{system-for-cross-domain-identity-management-core-schema}}

{[}IETF RFC 7643{]}~~~RFC 7643~~~September
2015~~~https://tools.ietf.org/html/rfc7643

\hypertarget{operations}{%
\subsection{Operations}\label{operations}}

Abstract: This article surveys the known standards concerning the
operation of identity systems.

\hypertarget{information-technology----security-techniques----a-framework-for-identity-management----part-3-practice-article-in-progress}{%
\subsubsection{\texorpdfstring{\emph{Information technology -\/- Security
techniques -\/- A framework for identity management -\/- Part 3:
Practice -- article in
progress}}{Information technology -\/- Security techniques -\/- A framework for identity management -\/- Part 3: Practice -- article in progress}}\label{information-technology----security-techniques----a-framework-for-identity-management----part-3-practice-article-in-progress}}

{[}ISO 24760-3{]}~~~ ISO/IEC 24760-3:2016 ~~~ 2016~~~
https://webstore.ansi.org/Standards/ISO/ISOIEC247602016~~~ \$162

\hypertarget{terminology}{%
\subsection{Terminology}\label{terminology}}

Abstract: This article surveys the known standards for the purpose of
collating and contrasting terminology defined.

\hypertarget{digital-identity-guidelines-1}{%
\subsubsection{\texorpdfstring{\emph{Digital Identity
Guidelines}}{Digital Identity Guidelines}}\label{digital-identity-guidelines-1}}

{[}SP 800-63-3{]}~~~ NIST Special Publication 800-63-3~~~ June 2017~~~
https://doi.org/10.6028/NIST.SP.800-63-3~~~

\hypertarget{an-ontology-of-identity-credentials-part-i-background-and-formulation}{%
\subsubsection{\texorpdfstring{\emph{An Ontology of Identity Credentials
Part I: Background and
Formulation}}{An Ontology of Identity Credentials Part I: Background and Formulation}}\label{an-ontology-of-identity-credentials-part-i-background-and-formulation}}

{[}SP 800-103{]}~~~ NIST Special Publication 800-103 (Draft)~~~ October
2006.~~~
https://tsapps.nist.gov/publication/get\_pdf.cfm?pub\_id=906227~~~

\hypertarget{security-and-privacy----a-framework-for-identity-management----part-1-terminology-and-concepts-article-in-progress}{%
\subsubsection{\texorpdfstring{\emph{Security and Privacy -\/- A Framework
For Identity Management -\/- Part 1: Terminology And Concepts -- article
in
progress}}{Security and Privacy -\/- A Framework For Identity Management -\/- Part 1: Terminology And Concepts -- article in progress}}\label{security-and-privacy----a-framework-for-identity-management----part-1-terminology-and-concepts-article-in-progress}}

{[}ISO 24760-1{]}~~~ ISO/IEC 24760-1:2019 IT ~~~ 2019~~~
https://webstore.ansi.org/Standards/ISO/ISOIEC247602019~~~ \$138

\hypertarget{isoiec-24760-12019-it-security-and-privacy----a-framework-for-identity-management----part-1-terminology-and-concepts}{%
\subsubsection{ISO/IEC 24760-1:2019 IT Security and Privacy -\/- A Framework
For Identity Management -\/- Part 1: Terminology And
Concepts}\label{isoiec-24760-12019-it-security-and-privacy----a-framework-for-identity-management----part-1-terminology-and-concepts}}

\hypertarget{emerging-societal-norms}{%
\section{Emerging Societal Norms}\label{emerging-societal-norms}}

\hypertarget{managing-consent-article-in-progress}{%
\subsection{\texorpdfstring{Managing Consent \emph{-- article in
progress}}{Managing Consent -- article in progress}}\label{managing-consent-article-in-progress}}

\hypertarget{workforce-iam-internal-iam}{%
\chapter{Workforce IAM / Internal
IAM}\label{workforce-iam-internal-iam}}

\hypertarget{iam-processes}{%
\section{IAM Processes}\label{iam-processes}}

\hypertarget{joiner-mover-leaver}{%
\subsection{Joiner-Mover-Leaver}\label{joiner-mover-leaver}}

\hypertarget{hr-ownership}{%
\subsection{HR Ownership}\label{hr-ownership}}

\hypertarget{provisioning-on-boarding-and-off-boarding}{%
\subsection{Provisioning (On-boarding and
Off-boarding)}\label{provisioning-on-boarding-and-off-boarding}}

\hypertarget{role-management}{%
\subsection{Role Management}\label{role-management}}

\hypertarget{re-certification}{%
\subsection{Re-certification}\label{re-certification}}

\hypertarget{compliance}{%
\section{Compliance}\label{compliance}}

\hypertarget{analytics-and-intelligence}{%
\section{Analytics and
Intelligence}\label{analytics-and-intelligence}}

\hypertarget{handling-business-partners-people}{%
\section{Handling Business Partners'
People}\label{handling-business-partners-people}}

\hypertarget{consumercitizen-iam}{%
\chapter{Consumer/Citizen IAM}\label{consumercitizen-iam}}

\hypertarget{ciam-vs-workforce-iam}{%
\section{CIAM vs Workforce IAM}\label{ciam-vs-workforce-iam}}

This introductory article reviews the main key differences between IAM
in the consumer world versus IAM in the enterprise. Some of these
differences include: focusing on the consumer experience and consumer
needs as opposed to the needs of the enterprise and offering a different
balance between what a consumer expects in terms of usability and
security versus enterprise requirements.

\hypertarget{consumer-journey}{%
\section{Consumer Journey}\label{consumer-journey}}

Consumers are the focus of the CIAM program. There are several areas
that need to be considered that could help you implement a successful
CIAM program, including the registration process for consumers,
determining and implementing assurance requirements, and the handling of
user consent. This section focuses on these areas, offering specific
examples and guidance for the IAM practitioner in the consumer-focused
industry.

\hypertarget{registration-of-consumers}{%
\subsection{Registration of
consumers}\label{registration-of-consumers}}

This article discusses consumer registration in a product or service.
Registration is one of the early experiences in your product. Too much
friction in this step would result in consumers going away. In general,
it's the idea of asking for as little as possible on first contact
(email-only or email+password registration) and then using various
profile enrichment strategies later on, e.g., MFA, shipping address,
phone number, etc.

\hypertarget{authentication-assurance-meeting-loa-requirements}{%
\subsection{Authentication assurance (meeting LoA
requirements)}\label{authentication-assurance-meeting-loa-requirements}}

Most activities in CIAM do not require a great level of assurance to be
able to do an operation, for example, updating a birthday or a display
name. This article explores the concept of levels of assurance (LoA) as
it applies to CIAM, including a review of activities that might require
a high authentication level of assurance as those are sensitive
activities such as the purchase of regulated goods, or access to
health-related records. In this case, another authentication process
might be rolled out, e.g., prompt another layer of authentication to
make sure the consumer is the right people perform the activities.~

\hypertarget{data-usage-consent}{%
\subsection{Data usage consent}\label{data-usage-consent}}

The consumer should know how his/her data is being used by the company
to give a better experience to the consumer. That's why it's important
to ask the consumer's consent to make sure they are all aware of their
data usage and store the consent to help with a dispute in case it
happens. This article references "Managing Consent" by Eve Maler and
Graham Williamson, currently in the BoK queue and focuses on additional
considerations specific to CIAM.

\hypertarget{social-sign-in-and-sign-up}{%
\subsection{Social sign-in and
sign-up}\label{social-sign-in-and-sign-up}}

Social sign-up offers a consumer a way to sign-up to a CIAM system that
takes advantage of existing accounts owned by the user. CIAM-focused
companies can effectively outsource some of the user support (such as
password management) to these social media systems and instead focus on
what information is required for personalization. This article explores
how social media logins can complement a CIAM infrastructure and offers
suggestions on how to offer the maximum benefit to the consumer. This
article ties closely to the Data Usage Content article.

\hypertarget{unified-consumer-view}{%
\section{Unified consumer view}\label{unified-consumer-view}}

This article describes the opportunities and challenges involved with
supporting a unified view of the consumers of a product or service to a
company in order to support targeted marketing, content, or product
recommendations. In order to have a unified consumer view, the CIAM
system could provide flexible attributes so the application is able to
add its own unique fields and help shape the consumer profile. Done
appropriately, this service can be of value to both the company and the
consumer.

\hypertarget{industry-considerations}{%
\section{Industry Considerations}\label{industry-considerations}}

\hypertarget{public-sector-vs-private-sector}{%
\subsection{Public sector vs private
sector}\label{public-sector-vs-private-sector}}

The article explains the unique use cases and challenges in the public
sector and private sector that should be considered by the IAM
practitioners. The article also provides the best practices and tips to
deal with the use cases and challenges. Almost every service requires a
different identification method in public sectors. Each governmental
agency has unique requirements for authentication. As an example
registering with your General Practitioners (GP) in the UK requires a
National Health Service number, while HMRC directs users to its
Government Gateway scheme to sign up and pay self-assessment taxes. This
net result is citizens need to have a variety of different
identification methods to complete straight forward tasks. The section
article explains tips and best practices for navigating this issue.

\hypertarget{strong-identity-proofing}{%
\subsubsection{Strong identity proofing}\label{strong-identity-proofing}}

Identity proofing is essential to enable the digital government. But the
extensive amount of data to prove the citizen identity has become one of
the challenges. The section explains the tips on navigating some of the
issues to create a strong yet consumer-friendly identity proofing.

\hypertarget{financial-services-the-section-explains-the-unique-use-cases-and-challenges-in-the-financial-industry-that-should-be-considered-by-iam-practitioners.-the-section-also-provides-the-best-practices-and-tips-to-deal-with-the-use-cases-and-challenges.}{%
\subsection{\texorpdfstring{Financial services\\
The section explains the unique use cases and challenges in the
financial industry that should be considered by IAM practitioners. The
section also provides the best practices and tips to deal with the use
cases and
challenges.}{Financial services The section explains the unique use cases and challenges in the financial industry that should be considered by IAM practitioners. The section also provides the best practices and tips to deal with the use cases and challenges.}}\label{financial-services-the-section-explains-the-unique-use-cases-and-challenges-in-the-financial-industry-that-should-be-considered-by-iam-practitioners.-the-section-also-provides-the-best-practices-and-tips-to-deal-with-the-use-cases-and-challenges.}}

\hypertarget{integration-with-the-legacy-system}{%
\subsubsection{Integration with the legacy
system}\label{integration-with-the-legacy-system}}

This should be considered given that most of the banks or financial
services have had their own system for a long time ago. Things like how
to let existing customers apply for new services easily should be
considered.

\hypertarget{high-level-of-assurance-on-sensitive-activities}{%
\subsubsection{High Level of Assurance on sensitive
activities}\label{high-level-of-assurance-on-sensitive-activities}}

Most of the activities in the financial services industry involve action
toward and accessing sensitive information, such as purchase goods,
funds transfer, etc. Due to this, there must be a high LoA to make sure
the right person performs the right activities. This article explores
ways of having a higher level of assurance and protects consumers from
fraud, e.g., perform step-up authentication, contextual authorization,
pin validation, card validation, etc.

\hypertarget{the-identities-delegation}{%
\subsubsection{The identities delegation}\label{the-identities-delegation}}

An example is a child managing a bank account on behalf of an elderly
parent. There are several challenges to deal with the use case. Some of
them are to deal with the power of attorney, and audit to make sure the
child doing things based on court authorization on behalf of the parent
and not just sharing the parent's password with the child. The article
explores the best practices to deal with the use case as it is becoming
more common use cases across several sectors, such as financial and
healthcare services.

\hypertarget{financial-regulations-compliance-and-guidance-from-the-government-organizations}{%
\subsubsection{Financial regulations compliance and guidance from the
government
organizations}\label{financial-regulations-compliance-and-guidance-from-the-government-organizations}}

There are specific regulations and organizational guidance in the
financial industry that help security and convenience to the consumer,
for example, Payment Service Directives 2 (PSD2), Open Banking,
Financial Ask Task Force organization. The article explains about those
and provides tips on how to comply with the regulation or follow the
organizational guidance.

\hypertarget{healthcare}{%
\subsection{Healthcare}\label{healthcare}}

The section explains the unique use cases and challenges in the
financial industry that should be considered by IAM practitioners. The
section also provides the best practices and tips to deal with those use
cases and challenges.

\hypertarget{high-level-of-assurance-on-sensitive-data-loa}{%
\subsubsection{High Level of Assurance on sensitive data
(LoA)}\label{high-level-of-assurance-on-sensitive-data-loa}}

Most data in the healthcare industry are sensitive data, e.g., a
patient's profile, disease history, medical records, etc., and so a high
level of assurance is required for making sure only the right person
accesses the right data.~There are several exceptions though. For
example, a homeless man who doesn't have a fixed address and no form of
authentication wants to access his data. The person deserves to access
his data but he can't prove himself. The section explains ways and best
practices for achieving the high LoA, e.g., step-authentication and to
deal with the unique use case such as the homeless man case, e.g.,
implements ``known to the practitioners'' or in other words the ability
of a practitioner (doctor) to vouch for the patient's identity

\hypertarget{identities-delegation}{%
\subsubsection{Identities delegation}\label{identities-delegation}}

An example is the parent and child relationship where the parent has
access to their child's medical records (provided consent was given).
There are several challenges to deal with the use case. Some of them are
to deal with the power of attorney and audit. The article explores the
best practices to deal with the use case as it is becoming more common
use cases across several sectors, such as financial and healthcare
services.

\hypertarget{healthcare-regulations-compliance}{%
\subsubsection{Healthcare regulations
compliance}\label{healthcare-regulations-compliance}}

The section explains the regulations in the healthcare industry such as
the Health Insurance Portability and Accountability Act (HIPAA) and the
tips to comply with those.

\hypertarget{game}{%
\subsection{Game}\label{game}}

\hypertarget{the-section-explains-unique-use-cases-and-challenges-faced-in-the-gaming-industry-that-should-be-considered-for-iam-practitioners.-the-section-also-explains-the-tips-and-best-practices-to-deal-with-those-use-cases-and-challenges.}{%
\subsection{The section explains unique use cases and challenges
faced in the gaming industry that should be considered for IAM
practitioners. The section also explains the tips and best practices to
deal with those use cases and
challenges.}\label{the-section-explains-unique-use-cases-and-challenges-faced-in-the-gaming-industry-that-should-be-considered-for-iam-practitioners.-the-section-also-explains-the-tips-and-best-practices-to-deal-with-those-use-cases-and-challenges.}}

\hypertarget{local-game-privacy-compliance}{%
\subsubsection{Local game privacy
compliance}\label{local-game-privacy-compliance}}

The section explains the regulations in the game industry that should be
considered while building CIAM such as General Data Protection
Regulation (GDPR) for EU players, and Shutdown Law for Korean players
and the tips to comply with those.

\hypertarget{scalability-and-availability}{%
\subsubsection{Scalability and availability
}\label{scalability-and-availability}}

There are around 1.2 billion players in the world. Knowing this, the
scalability and the high availability are important factors for having a
successful CIAM. The article explains the tips and best practices to
handle the load and keep the game services online at all times.~

\hypertarget{gaming-and-authentication}{%
\subsubsection{Gaming and authentication}\label{gaming-and-authentication}}

Most mobile games do not require authentication at the start so the
player could start playing immediately thus increasing the player
engagement. This could be achieved by creating an anonymous account at
the start of the game. The article explores the tips to deal with this
``expectation'', anonymous account implementation, and account upgrade
implementation to help players secure their account.

\hypertarget{privacy-and-compliance}{%
\section{Privacy and Compliance}\label{privacy-and-compliance}}

Privacy and compliance capabilities are foundational and the CIAM
program should focus on protecting the individual. CIAM teams must
adhere to an increasing number of consumer protection laws and
regulations. For example the EU General Data Protection Regulation
(GDPR) and California Consumer Privacy Act (CCPA). Multinational
companies should worry about the privacy compliance of each and every
country they do business with. This article builds on other areas of the
BoK that consider specific regulations like GDPR and discusses the
specific considerations of privacy and compliance in a consumer-focused
environment.

\hypertarget{security}{%
\section{Security}\label{security}}

Good security must underpin all CIAM initiatives as this is the key to
protect consumer data and to maintain their trust in our system. It is
important to remember user experience should be considered as well while
creating a good security model. The following sections explain some key
methods for achieving good security.

\hypertarget{adaptive-authentication}{%
\subsection{Adaptive authentication}\label{adaptive-authentication}}

It is an authentication action that takes account of other
dynamic-runtime environment data or context-based attributes, e.g.,
device location, time to login, etc., in addition to credentials such as
username and password to authenticate users. The authentication is also
known as risk-based or contextual authentication.

\hypertarget{multi-factor-authentication-mfa}{%
\subsection{Multi-Factor Authentication
(MFA)}\label{multi-factor-authentication-mfa}}

This refers to the use of more than one credential in the authentication
of the user. Generally, the use of multiple factors results in a higher
LoA for the user's authentication. Two-factor (2FA) is the simplest
example of MFA where two different credentials are used. MFA provides a
variety of factors to choose from, ranging from asking a security
question to capturing and confirming biometric data to using physical
authentication keys, codes or One-Time Passwords (OTPs) over SMS/email
or Time-based One-time Password (TOTP) (Google Authenticator).~

\hypertarget{non-human-entity}{%
\chapter{Non-Human Entity}\label{non-human-entity}}

\hypertarget{operational-technology-ot}{%
\section{Operational Technology
(OT)}\label{operational-technology-ot}}

\hypertarget{iot-devices}{%
\section{IoT Devices}\label{iot-devices}}

\hypertarget{iot-sectors}{%
\subsection{IoT Sectors}\label{iot-sectors}}

\hypertarget{home-automation}{%
\subsubsection{Home Automation}\label{home-automation}}

\hypertarget{personal-wearables}{%
\subsubsection{Personal (wearables)}\label{personal-wearables}}

\hypertarget{implants}{%
\subsubsection{Implants}\label{implants}}

\hypertarget{plant-automation}{%
\subsubsection{Plant Automation}\label{plant-automation}}

\hypertarget{vehicle}{%
\subsubsection{Vehicle}\label{vehicle}}

\hypertarget{smart-cities}{%
\subsubsection{Smart Cities}\label{smart-cities}}

\hypertarget{agriculture}{%
\subsubsection{Agriculture}\label{agriculture}}

\hypertarget{buildingindustrial}{%
\subsubsection{Building/Industrial}\label{buildingindustrial}}

\hypertarget{utilities}{%
\subsubsection{Utilities}\label{utilities}}

\hypertarget{rpa-robotics}{%
\section{RPA / robotics}\label{rpa-robotics}}

\hypertarget{security-requirements}{%
\section{Security requirements}\label{security-requirements}}

\hypertarget{iam-architecture-and-solutions}{%
\chapter{IAM Architecture and Solutions
}\label{iam-architecture-and-solutions}}

\hypertarget{iam-architecture-overview-article-in-progress}{%
\section{\texorpdfstring{IAM Architecture Overview \emph{-- article
in
progress}}{IAM Architecture Overview -- article in progress}}\label{iam-architecture-overview-article-in-progress}}

\hypertarget{architecture-patterns-article-in-progress}{%
\section{\texorpdfstring{Architecture Patterns \emph{-- article in
progress}}{Architecture Patterns -- article in progress}}\label{architecture-patterns-article-in-progress}}

\hypertarget{technical-architecture-article-in-progress}{%
\section{\texorpdfstring{Technical Architecture \emph{-- article in
progress}}{Technical Architecture -- article in progress}}\label{technical-architecture-article-in-progress}}

\hypertarget{identity-governance-article-in-progress}{%
\section{\texorpdfstring{Identity Governance \emph{-- article in
progress}}{Identity Governance -- article in progress}}\label{identity-governance-article-in-progress}}

\hypertarget{elements-of-iga-systems-article-in-progress}{%
\subsection{\texorpdfstring{Elements of IGA Systems \emph{-- article
in
progress}}{Elements of IGA Systems -- article in progress}}\label{elements-of-iga-systems-article-in-progress}}

\hypertarget{key-definitions-and-terms-article-in-progress}{%
\section{\texorpdfstring{Key Definitions and Terms \emph{-- article
in
progress}}{Key Definitions and Terms -- article in progress}}\label{key-definitions-and-terms-article-in-progress}}

\hypertarget{business-system}{%
\section{Business System}\label{business-system}}

\hypertarget{business-processes}{%
\subsection{Business Processes}\label{business-processes}}

\hypertarget{recertification-of-accounts}{%
\subsubsection{Recertification of
accounts}\label{recertification-of-accounts}}

\hypertarget{recommended-practices}{%
\section{Recommended Practices}\label{recommended-practices}}

\hypertarget{design-for-security}{%
\subsection{Design for security}\label{design-for-security}}

\hypertarget{operational-considerations}{%
\chapter{Operational Considerations}\label{operational-considerations}}

\hypertarget{account-recovery}{%
\section{Account recovery}\label{account-recovery}}

\hypertarget{call-centers}{%
\section{Call centers}\label{call-centers}}

\hypertarget{engagement-of-user-for-their-own-security}{%
\section{Engagement of user for their own
security}\label{engagement-of-user-for-their-own-security}}

\hypertarget{security-events-and-operations}{%
\section{Security events and
operations}\label{security-events-and-operations}}

\hypertarget{project-management}{%
\chapter{Project Management}\label{project-management}}

Many Identity and Access Management (IAM) projects proceed without a
project manager. In these cases the IT group in charge of identity
management are left to deploy the required solution in the absence of
any overarching management. While this is sometimes seen as the most
expedient way to get a system installed or updated, it is short-sighted
and likely to cost the organisation more money in the longer term. An
IAM solution touches so many systems within an organisation and is
dependent on the current and planned condition of so many applications
that to deploy a solution without properly considering the impact,
managing the required resources and keeping management advised of
progress, will result in a substandard deployment.

Here we look at two ways to manage a project -- ``Classic'', sometimes
called Waterfall, and ``Agile, a way to manage projects that
accommodates changes that inevitably arise during the course of a
project.

Reference is made to the Project Management Institute (PMI) Framework.
This document in no way seeks to replicate the PMI's methodology or
replace the project management training that the PMI provides. The
reader is referred to the PMI Body of Knowledge for further information.

\hypertarget{project-management-institute-framework-and-project-management-office-issues-article-in-progress}{%
\section{\texorpdfstring{Project Management Institute Framework and
Project Management Office Issues \emph{-- article in
progress}}{Project Management Institute Framework and Project Management Office Issues -- article in progress}}\label{project-management-institute-framework-and-project-management-office-issues-article-in-progress}}

\hypertarget{new-implementation-projects}{%
\section{New Implementation
Projects}\label{new-implementation-projects}}

\hypertarget{migration-projects}{%
\section{Migration Projects}\label{migration-projects}}

\hypertarget{iam-knowledge-sharing}{%
\chapter{IAM Knowledge Sharing}\label{iam-knowledge-sharing}}

\hypertarget{independent-organizations-articles-in-progress}{%
\section{\texorpdfstring{Independent Organizations \emph{-- articles
in
progress}}{Independent Organizations -- articles in progress}}\label{independent-organizations-articles-in-progress}}

\hypertarget{standards-bodies}{%
\section{Standards Bodies}\label{standards-bodies}}

\hypertarget{analyst-organizations}{%
\section{Analyst Organizations}\label{analyst-organizations}}

\hypertarget{conferences}{%
\section{Conferences}\label{conferences}}

\hypertarget{advanced-topics-parking-lot}{%
\chapter{Advanced Topics -- Parking
Lot}\label{advanced-topics-parking-lot}}

\hypertarget{digital-legacy---handling-deceased-persons-digital-id-advanced-topic}{%
\section{Digital Legacy - handling deceased persons' digital ID
(Advanced
Topic)}\label{digital-legacy---handling-deceased-persons-digital-id-advanced-topic}}

\hypertarget{self-sovereign-identity}{%
\section{Self-Sovereign Identity}\label{self-sovereign-identity}}

\hypertarget{blockchain-id}{%
\subsection{Blockchain ID}\label{blockchain-id}}

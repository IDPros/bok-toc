\hypertarget{introduction}{%
\chapter{Introduction}\label{introduction}}

\hypertarget{ethics}{%
\section{Ethics}\label{ethics}}

\hypertarget{information-security}{%
\section{Information security}\label{information-security}}

\hypertarget{trust-say-more---what-is-this}{%
\subsection{Trust (say more - what is
this?)}\label{trust-say-more---what-is-this}}

\hypertarget{privacy}{%
\section{Privacy}\label{privacy}}

\hypertarget{identification-and-authentication}{%
\section{Identification and
authentication}\label{identification-and-authentication}}

\hypertarget{context-and-identity}{%
\subsection{Context and Identity}\label{context-and-identity}}

\hypertarget{levels-of-assurance}{%
\subsection{Levels of Assurance}\label{levels-of-assurance}}

\hypertarget{the-business-case-for-iam}{%
\section{The Business Case for IAM}\label{the-business-case-for-iam}}

\hypertarget{workforce-iam}{%
\subsection{Workforce IAM}\label{workforce-iam}}

\hypertarget{consumercitizen-iam}{%
\subsection{Consumer/Citizen IAM}\label{consumercitizen-iam}}

\hypertarget{digital-identity}{%
\chapter{Digital Identity}\label{digital-identity}}

\hypertarget{definition}{%
\section{Definition}\label{definition}}

\hypertarget{reputation}{%
\subsection{Reputation}\label{reputation}}

\hypertarget{laws-of-identity-this-sounds-like-jurisdictions-and-real-laws---is-that-the-intent}{%
\subsection{Laws of Identity (this sounds like jurisdictions and real
laws - is that the
intent?)}\label{laws-of-identity-this-sounds-like-jurisdictions-and-real-laws---is-that-the-intent}}

\hypertarget{identifiers}{%
\section{Identifiers}\label{identifiers}}

\hypertarget{digital-identity-lifecycle}{%
\section{Digital Identity Lifecycle
(?)}\label{digital-identity-lifecycle}}

\hypertarget{mapping-to-human-or-device}{%
\section{Mapping to human or
device}\label{mapping-to-human-or-device}}

\hypertarget{proofing-binding-or-registration}{%
\section{Proofing, Binding or Registration
(?)}\label{proofing-binding-or-registration}}

\hypertarget{verificationvalidation}{%
\subsection{Verification/Validation}\label{verificationvalidation}}

\hypertarget{credentials}{%
\section{Credentials}\label{credentials}}

\hypertarget{access-control}{%
\chapter{Access Control}\label{access-control}}

\hypertarget{authentication}{%
\section{Authentication}\label{authentication}}

\hypertarget{dynamic-authentication-risk-based}{%
\subsection{Dynamic Authentication
(risk-based)}\label{dynamic-authentication-risk-based}}

\hypertarget{multi-factor-authentication}{%
\subsection{Multi-Factor
Authentication}\label{multi-factor-authentication}}

\hypertarget{single-sign-on-within-a-domain}{%
\subsection{Single Sign-on Within a
Domain}\label{single-sign-on-within-a-domain}}

\hypertarget{centralised-authentication-service}{%
\subsection{Centralised Authentication
Service}\label{centralised-authentication-service}}

\hypertarget{federated-authentication-between-domains}{%
\subsection{Federated Authentication (between
domains)}\label{federated-authentication-between-domains}}

\hypertarget{device-identity-for-corroboration}{%
\subsection{Device Identity for
Corroboration}\label{device-identity-for-corroboration}}

\hypertarget{fast-identity-online-fido-and-its-cousins}{%
\subsection{Fast Identity Online (FIDO) and its
cousins}\label{fast-identity-online-fido-and-its-cousins}}

\hypertarget{session-management}{%
\subsection{Session Management}\label{session-management}}

\hypertarget{authorization}{%
\section{Authorization}\label{authorization}}

\hypertarget{resources-to-protect}{%
\subsection{Resources to Protect}\label{resources-to-protect}}

\hypertarget{authorisation}{%
\subsection{Authorisation}\label{authorisation}}

\hypertarget{acls}{%
\subsubsection{ACL's}\label{acls}}

\hypertarget{rbac}{%
\subsubsection{RBAC}\label{rbac}}

\hypertarget{abac-dynamic-access-management}{%
\subsubsection{ABAC / Dynamic Access
Management}\label{abac-dynamic-access-management}}

\hypertarget{policy-management-solutions}{%
\paragraph{Policy Management
solutions}\label{policy-management-solutions}}

\hypertarget{privileged-access-management}{%
\subsection{Privileged Access
Management}\label{privileged-access-management}}

\hypertarget{alignment-to-risk-management}{%
\subsubsection{Alignment to Risk
Management}\label{alignment-to-risk-management}}

\hypertarget{system-accounts}{%
\subsubsection{System Accounts}\label{system-accounts}}

\hypertarget{laws-regulations-and-standards}{%
\chapter{Laws, Regulations, and
Standards}\label{laws-regulations-and-standards}}

Abstract: This chapter provides information about the externally defined
environment in which Identity and Access management professionals
operate.~ The laws are documents that define duties and consequences in
legal jurisdictions, such as countries. Regulations are more specific
and detailed requirements.~ Standards may also be mandatory; government
entities often require compliance with standards produced by certain
standards bodies. We also include \emph{de facto} standards and
recommended practices here.

\hypertarget{framework-to-understand-legal-environment}{%
\section{Framework to Understand Legal
Environment}\label{framework-to-understand-legal-environment}}

Abstract: Identity systems and its participants are governed by a myriad
and complex set of laws, regulations, and contractual requirements, and
the obligations they impose are not always clear. This article focuses
on the legal environment that governs identity systems.~ The emphasis is
on United States, but references are made to other countries' laws and
efforts to coordinate rules underway in the UN Commission on
International Trade Law (UNCITRAL) regarding identity management
legislation.

\hypertarget{highlights-of-selected-laws}{%
\section{Highlights of Selected
Laws}\label{highlights-of-selected-laws}}

Abstract: This section is organized by jurisdiction.~ It is intended to
provide at a minimum a reference to known laws and regulations in
jurisdictions likely to be encountered by our membership.~ At present
this includes Europe, United States, and Canada will likely also include
Australia in the short term.

\hypertarget{europe}{%
\subsection{Europe}\label{europe}}

\hypertarget{gdpr}{%
\subsubsection{GDPR}\label{gdpr}}

Abstract: This article provides a basic understanding of how the
\emph{General Data Protection Regulation (GDPR)} applies when processing
`any information relating to an identified or identifiable natural
person'.

\hypertarget{united-states}{%
\subsection{United States}\label{united-states}}

Abstract:~ This article explains how identity and access management
supports the requirements of prominent U.S. laws.

\hypertarget{sarbanes-oxley-section-404}{%
\subsubsection{Sarbanes-Oxley Section
404}\label{sarbanes-oxley-section-404}}

\hypertarget{health-insurance-portability-and-accountability-act-hipaa}{%
\subsubsection{Health Insurance Portability and Accountability Act
(HIPAA)}\label{health-insurance-portability-and-accountability-act-hipaa}}

\hypertarget{health-information-technology-for-economic-and-clinical-health-act-hitech}{%
\subsubsection{Health Information Technology for Economic and Clinical
Health Act
(HITECH)}\label{health-information-technology-for-economic-and-clinical-health-act-hitech}}

\hypertarget{family-educational-rights-and-privacy-act-of-1974-ferpa}{%
\subsubsection{Family Educational Rights and Privacy Act of 1974
(FERPA)}\label{family-educational-rights-and-privacy-act-of-1974-ferpa}}

\hypertarget{childrens-online-privacy-protection-act-coppa}{%
\subsubsection{Children's Online Privacy Protection Act
(COPPA)}\label{childrens-online-privacy-protection-act-coppa}}

\hypertarget{fair-and-accurate-credit-transaction-act-facta}{%
\subsubsection{Fair and Accurate Credit Transaction Act
(FACTA)}\label{fair-and-accurate-credit-transaction-act-facta}}

\hypertarget{canada}{%
\subsection{Canada}\label{canada}}

Abstract:~ This article explains how identity and access management
support the requirements of prominent Canadian laws.

\hypertarget{personal-information-protection-and-electronic-documents-act-piped-act-or-pipeda}{%
\subsubsection{Personal Information Protection and Electronic Documents Act
(PIPED Act, or
PIPEDA)}\label{personal-information-protection-and-electronic-documents-act-piped-act-or-pipeda}}

\hypertarget{regulations}{%
\section{Regulations}\label{regulations}}

Abstract:~ This article explains how identity and access management
supports the requirements of prominent regulations.

\hypertarget{standards}{%
\section{Standards}\label{standards}}

Abstract: There are many standards. Standards may be mandatory such as
when government entities require compliance with standards produced by
certain standards bodies.~ We also include \emph{de facto} standards and
recommended practices here. This is a curated set of standards that have
been deemed to be useful to identity professionals.~ They are organized
topically, not by their source. Standards that span more than one topic
are possible. In this case cross references may be used.

\hypertarget{architecture}{%
\subsection{Architecture}\label{architecture}}

Abstract: This article surveys the known standards concerning
architecture for identity systems.

\hypertarget{isoiec-24760-22015-information-technology----security-techniques----a-framework-for-identity-management----part-2-reference-architecture-and-requirements}{%
\subsubsection{ISO/IEC 24760-2:2015 Information technology -\/- Security
techniques -\/- A framework for identity management -\/- Part 2:
Reference architecture and
requirements}\label{isoiec-24760-22015-information-technology----security-techniques----a-framework-for-identity-management----part-2-reference-architecture-and-requirements}}

\hypertarget{assurance}{%
\subsection{Assurance}\label{assurance}}

Abstract: This article surveys the known standards concerning risk and
assurance for identity systems.

\hypertarget{standard-on-identity-and-credential-assurance}{%
\subsubsection{\texorpdfstring{\emph{Standard on Identity and Credential
Assurance}}{Standard on Identity and Credential Assurance}}\label{standard-on-identity-and-credential-assurance}}

{[}Canada{]}~~~ Government of Canada~~~ February 2013~~~
https://www.tbs-sct.gc.ca/pol/doc-eng.aspx?id=26776.~~~ Archived - Need
successors

\hypertarget{digital-identity-guidelines}{%
\subsubsection{\texorpdfstring{\emph{Digital Identity
Guidelines}}{Digital Identity Guidelines}}\label{digital-identity-guidelines}}

{[}SP 800-63-3{]}~~~ NIST Special Publication 800-63-3~~~ June 2017~~~
https://doi.org/10.6028/NIST.SP.800-63-3~~~

\hypertarget{guide-for-applying-the-risk-management-framework-to-federal-information-systems-a-security-life-cycle-approach}{%
\subsubsection{\texorpdfstring{\emph{Guide for Applying the Risk Management
Framework to Federal Information Systems: A Security Life Cycle
Approach}}{Guide for Applying the Risk Management Framework to Federal Information Systems: A Security Life Cycle Approach}}\label{guide-for-applying-the-risk-management-framework-to-federal-information-systems-a-security-life-cycle-approach}}

{[}SP-800-37{]}~~~ NIST Special Publication 800-37r1~~~ June 2014~~~
https://doi.org/10.6028/NIST.SP.800-37r1~~~

\hypertarget{authentication-1}{%
\subsection{Authentication}\label{authentication-1}}

Abstract: This article surveys the known standards concerning methods of
authenticating principals.

\hypertarget{digital-identity-guidelines-authentication-and-lifecycle-management}{%
\subsubsection{\texorpdfstring{\emph{Digital Identity Guidelines:
Authentication and Lifecycle
Management}}{Digital Identity Guidelines: Authentication and Lifecycle Management}}\label{digital-identity-guidelines-authentication-and-lifecycle-management}}

{[}SP 800-63B{]}~~~ NIST Special Publication 800-63C~~~ December 2017~~~
https://doi.org/10.6028/NIST.SP.800-63b~~~

\hypertarget{introduction-to-public-key-technology-and-the-federal-pki-infrastructure}{%
\subsubsection{\texorpdfstring{\emph{Introduction to Public Key Technology
and the Federal PKI
Infrastructure}}{Introduction to Public Key Technology and the Federal PKI Infrastructure}}\label{introduction-to-public-key-technology-and-the-federal-pki-infrastructure}}

{[}SP 800-32{]}~~~ NIST Special Publication 800-32~~~ February 2001.~~~
https://tsapps.nist.gov/publication/get\_pdf.cfm?pub\_id=151247~~~

\hypertarget{lightweight-directory-access-protocol-ldap-technical-specification-road-map}{%
\subsubsection{\texorpdfstring{\emph{Lightweight Directory Access Protocol
(LDAP): Technical Specification Road
Map}}{Lightweight Directory Access Protocol (LDAP): Technical Specification Road Map}}\label{lightweight-directory-access-protocol-ldap-technical-specification-road-map}}

{[}IETF RFC 4510{]}~~~ RFC 4510~~~ June 2006~~~
https://tools.ietf.org/html/rfc4510~~~

\hypertarget{openid-connect-core-1.0-incorporating-errata-set-1}{%
\subsubsection{\texorpdfstring{\emph{OpenID Connect Core 1.0 incorporating
errata set
1}}{OpenID Connect Core 1.0 incorporating errata set 1}}\label{openid-connect-core-1.0-incorporating-errata-set-1}}

{[}OIDC{]}~~~ Sakimura, N., Bradley, B., Jones, M., de Medeiros, B., and
C. Mortimore~~~ November 2014~~~
https://openid.net/specs/openid-connect-core-1\_0.html.~~~

\hypertarget{personal-identity-verification-piv-of-federal-employees-and-contractors}{%
\subsubsection{\texorpdfstring{\emph{Personal Identity Verification (PIV) of
Federal Employees and
Contractors}}{Personal Identity Verification (PIV) of Federal Employees and Contractors}}\label{personal-identity-verification-piv-of-federal-employees-and-contractors}}

{[}FIPS 201-2{]}~~~ NIST FIPS Publication 201-2~~~ September 2013~~~
https://doi.org/10.6028/NIST.FIPS.201-2~~~

\hypertarget{biometric-data-specification-for-personal-identity-verification}{%
\subsubsection{\texorpdfstring{\emph{Biometric Data Specification for
Personal Identity
Verification}}{Biometric Data Specification for Personal Identity Verification}}\label{biometric-data-specification-for-personal-identity-verification}}

{[}SP 800-76-2{]}~~~ NIST Special Publication 800-76-2~~~ July 2013~~~
https://doi.org/10.6028/NIST.SP.800-76-2~~~

\hypertarget{authorization-1}{%
\subsection{Authorization}\label{authorization-1}}

Abstract: This article surveys the known standards concerning methods of
access control. These standards involve protecting resources.~ This is
sometimes called authorization.

\hypertarget{the-oauth-2.0-authorization-framework}{%
\subsubsection{\texorpdfstring{\emph{The OAuth 2.0 Authorization
Framework}}{The OAuth 2.0 Authorization Framework}}\label{the-oauth-2.0-authorization-framework}}

{[}IETF RFC 6749{]}~~~ RFC 6749~~~ October 2012~~~
https://tools.ietf.org/html/rfc6749~~~

\hypertarget{user-managed-access-uma-profile-of-oauth-2.0}{%
\subsubsection{\texorpdfstring{\emph{User-Managed Access (UMA) Profile of
OAuth
2.0}}{User-Managed Access (UMA) Profile of OAuth 2.0}}\label{user-managed-access-uma-profile-of-oauth-2.0}}

Abstract: The weaknesses of many notice-and-consent paradigms of data
privacy are clear. This article notes the social, legal and regulatory
drivers and examines some approaches to satisfy them.

{[}KI UMA{]}~~~ Kantara Initiative UMA Recommendation~~~ December
2015~~~ https://docs.kantarainitiative.org/uma/rec-uma-core.html~~~

\hypertarget{federation}{%
\subsection{Federation}\label{federation}}

Abstract: This article surveys the known standards concerning methods of
allowing authentication from one domain to be honored in another.

\hypertarget{openid-connect-core-1.0-incorporating-errata-set-1-1}{%
\subsubsection{\texorpdfstring{\emph{OpenID Connect Core 1.0 incorporating
errata set
1}}{OpenID Connect Core 1.0 incorporating errata set 1}}\label{openid-connect-core-1.0-incorporating-errata-set-1-1}}

{[}OIDC{]}~~~ Sakimura, N., Bradley, B., Jones, M., de Medeiros, B., and
C. Mortimore~~~ November 2014~~~
https://openid.net/specs/openid-connect-core-1\_0.html.~~~

\hypertarget{assertions-and-protocols-for-the-oasis-security-assertion-markup-language-saml-v2.0}{%
\subsubsection{\texorpdfstring{\emph{Assertions and Protocols for the OASIS
Security Assertion Markup Language (SAML)
V2.0}}{Assertions and Protocols for the OASIS Security Assertion Markup Language (SAML) V2.0}}\label{assertions-and-protocols-for-the-oasis-security-assertion-markup-language-saml-v2.0}}

{[}OASIS SAML 2{]}~~~ SAML 2.0~~~ March 2005~~~
http://docs.oasis-open.org/security/saml/v2.0/saml-core-2.0-os.pdf~~~

\hypertarget{digital-identity-guidelines-federation-and-assertions}{%
\subsubsection{\texorpdfstring{\emph{Digital Identity Guidelines: Federation
and
Assertions}}{Digital Identity Guidelines: Federation and Assertions}}\label{digital-identity-guidelines-federation-and-assertions}}

{[}SP 800-63C{]}~~~ NIST Special Publication 800-63C~~~ December 2017~~~
https://doi.org/10.6028/NIST.SP.800-63c~~~

\hypertarget{lifecycle}{%
\subsection{Lifecycle}\label{lifecycle}}

Abstract: This article surveys the known standards concerning the
creation and registration of identities and subsequent changes to the
characteristics of those identities and the eventual removal of the
same.

\hypertarget{standard-on-identity-and-credential-assurance-1}{%
\subsubsection{\texorpdfstring{\emph{Standard on Identity and Credential
Assurance}}{Standard on Identity and Credential Assurance}}\label{standard-on-identity-and-credential-assurance-1}}

{[}Canada{]}~~~ Government of Canada~~~ February 2013~~~
https://www.tbs-sct.gc.ca/pol/doc-eng.aspx?id=26776.~~~ Archived - Need
successors

\hypertarget{digital-identity-guidelines-enrollment-and-identity-proofing-requirements}{%
\subsubsection{\texorpdfstring{\emph{Digital Identity Guidelines: Enrollment
and Identity Proofing
Requirements}}{Digital Identity Guidelines: Enrollment and Identity Proofing Requirements}}\label{digital-identity-guidelines-enrollment-and-identity-proofing-requirements}}

{[}SP 800-63A{]}~~~ NIST Special Publication 800-63A~~~ December 2017~~~
https://doi.org/10.6028/NIST.SP.800-63a~~~

\hypertarget{digital-identity-guidelines-authentication-and-lifecycle-management-1}{%
\subsubsection{\texorpdfstring{\emph{Digital Identity Guidelines:
Authentication and Lifecycle
Management}}{Digital Identity Guidelines: Authentication and Lifecycle Management}}\label{digital-identity-guidelines-authentication-and-lifecycle-management-1}}

{[}SP 800-63B{]}~~~ NIST Special Publication 800-63C~~~ December 2017~~~
https://doi.org/10.6028/NIST.SP.800-63b~~~

\hypertarget{operations}{%
\subsection{Operations}\label{operations}}

Abstract: This article surveys the known standards concerning the
operation of identity systems.

\hypertarget{information-technology----security-techniques----a-framework-for-identity-management----part-3-practice}{%
\subsubsection{\texorpdfstring{\emph{Information technology -\/- Security
techniques -\/- A framework for identity management -\/- Part 3:
Practice}}{Information technology -\/- Security techniques -\/- A framework for identity management -\/- Part 3: Practice}}\label{information-technology----security-techniques----a-framework-for-identity-management----part-3-practice}}

{[}ISO 24760-3{]}~~~ ISO/IEC 24760-3:2016 ~~~ 2016~~~
https://webstore.ansi.org/Standards/ISO/ISOIEC247602016~~~ \$162

\hypertarget{terminology}{%
\subsection{Terminology}\label{terminology}}

Abstract: This article surveys the known standards for the purpose of
collating and contrasting terminology defined.

\hypertarget{digital-identity-guidelines-1}{%
\subsubsection{\texorpdfstring{\emph{Digital Identity
Guidelines}}{Digital Identity Guidelines}}\label{digital-identity-guidelines-1}}

{[}SP 800-63-3{]}~~~ NIST Special Publication 800-63-3~~~ June 2017~~~
https://doi.org/10.6028/NIST.SP.800-63-3~~~

\hypertarget{an-ontology-of-identity-credentials-part-i-background-and-formulation}{%
\subsubsection{\texorpdfstring{\emph{An Ontology of Identity Credentials
Part I: Background and
Formulation}}{An Ontology of Identity Credentials Part I: Background and Formulation}}\label{an-ontology-of-identity-credentials-part-i-background-and-formulation}}

{[}SP 800-103{]}~~~ NIST Special Publication 800-103 (Draft)~~~ October
2006.~~~
https://tsapps.nist.gov/publication/get\_pdf.cfm?pub\_id=906227~~~

\hypertarget{security-and-privacy----a-framework-for-identity-management----part-1-terminology-and-concepts}{%
\subsubsection{\texorpdfstring{\emph{Security and Privacy -\/- A Framework
For Identity Management -\/- Part 1: Terminology And
Concepts}}{Security and Privacy -\/- A Framework For Identity Management -\/- Part 1: Terminology And Concepts}}\label{security-and-privacy----a-framework-for-identity-management----part-1-terminology-and-concepts}}

{[}ISO 24760-1{]}~~~ ISO/IEC 24760-1:2019 IT ~~~ 2019~~~
https://webstore.ansi.org/Standards/ISO/ISOIEC247602019~~~ \$138

\hypertarget{isoiec-24760-12019-it-security-and-privacy----a-framework-for-identity-management----part-1-terminology-and-concepts}{%
\subsubsection{ISO/IEC 24760-1:2019 IT Security and Privacy -\/- A Framework
For Identity Management -\/- Part 1: Terminology And
Concepts}\label{isoiec-24760-12019-it-security-and-privacy----a-framework-for-identity-management----part-1-terminology-and-concepts}}

\hypertarget{workforce-iam-internal-iam}{%
\chapter{Workforce IAM / Internal
IAM}\label{workforce-iam-internal-iam}}

\hypertarget{iam-processes}{%
\section{IAM Processes}\label{iam-processes}}

\hypertarget{joiner-mover-leaver}{%
\subsection{Joiner-Mover-Leaver}\label{joiner-mover-leaver}}

\hypertarget{hr-ownership}{%
\subsection{HR Ownership}\label{hr-ownership}}

\hypertarget{provisioning-on-boarding-and-off-boarding}{%
\subsection{Provisioning (On-boarding and
Off-boarding)}\label{provisioning-on-boarding-and-off-boarding}}

\hypertarget{role-management}{%
\subsection{Role Management}\label{role-management}}

\hypertarget{re-certification}{%
\subsection{Re-certification}\label{re-certification}}

\hypertarget{compliance}{%
\section{Compliance}\label{compliance}}

\hypertarget{analytics-and-intelligence}{%
\section{Analytics and
Intelligence}\label{analytics-and-intelligence}}

\hypertarget{handling-business-partners-people}{%
\section{Handling Business Partners'
People}\label{handling-business-partners-people}}

\hypertarget{consumercitizen-iam-1}{%
\chapter{Consumer/Citizen IAM}\label{consumercitizen-iam-1}}

\hypertarget{consumer-journey-identification-to-loyal-customer}{%
\section{Consumer Journey (identification to loyal
customer)}\label{consumer-journey-identification-to-loyal-customer}}

\hypertarget{registration-of-consumers}{%
\subsection{Registration of
Consumers}\label{registration-of-consumers}}

\hypertarget{authentication-assurance-meeting-loa-requirements}{%
\subsection{Authentication Assurance (meeting LoA
requirements)}\label{authentication-assurance-meeting-loa-requirements}}

\hypertarget{industry-considerations}{%
\section{Industry Considerations}\label{industry-considerations}}

\hypertarget{public-sector-vs.-private-sector}{%
\subsection{Public Sector vs.~Private
Sector}\label{public-sector-vs.-private-sector}}

\hypertarget{financial-services}{%
\subsection{Financial Services}\label{financial-services}}

\hypertarget{healthcare}{%
\subsection{Healthcare}\label{healthcare}}

\hypertarget{social-sign-up-and-sign-on}{%
\section{Social Sign-up and
Sign-on}\label{social-sign-up-and-sign-on}}

\hypertarget{non-human-entity}{%
\chapter{Non-Human Entity}\label{non-human-entity}}

\hypertarget{operational-technology-ot}{%
\section{Operational Technology
(OT)}\label{operational-technology-ot}}

\hypertarget{iot-devices}{%
\section{IoT Devices}\label{iot-devices}}

\hypertarget{iot-sectors}{%
\subsection{IoT Sectors}\label{iot-sectors}}

\hypertarget{home-automation}{%
\subsubsection{Home Automation}\label{home-automation}}

\hypertarget{personal-wearables}{%
\subsubsection{Personal (wearables)}\label{personal-wearables}}

\hypertarget{implants}{%
\subsubsection{Implants}\label{implants}}

\hypertarget{plant-automation}{%
\subsubsection{Plant Automation}\label{plant-automation}}

\hypertarget{vehicle}{%
\subsubsection{Vehicle}\label{vehicle}}

\hypertarget{smart-cities}{%
\subsubsection{Smart Cities}\label{smart-cities}}

\hypertarget{agriculture}{%
\subsubsection{Agriculture}\label{agriculture}}

\hypertarget{buildingindustrial}{%
\subsubsection{Building/Industrial}\label{buildingindustrial}}

\hypertarget{utilities}{%
\subsubsection{Utilities}\label{utilities}}

\hypertarget{rpa-robotics}{%
\section{RPA / robotics}\label{rpa-robotics}}

\hypertarget{security-requirements}{%
\section{Security requirements}\label{security-requirements}}

\hypertarget{iam-architecture-and-solutions}{%
\chapter{IAM Architecture and
Solutions}\label{iam-architecture-and-solutions}}

\hypertarget{business-system}{%
\section{Business System}\label{business-system}}

\hypertarget{business-processes}{%
\subsection{Business Processes}\label{business-processes}}

\hypertarget{recertification-of-accounts}{%
\subsubsection{Recertification of
accounts}\label{recertification-of-accounts}}

\hypertarget{informationdata-architecture}{%
\section{Information/Data
Architecture}\label{informationdata-architecture}}

\hypertarget{application-portfolio}{%
\section{Application Portfolio}\label{application-portfolio}}

\hypertarget{apis}{%
\subsection{APIs}\label{apis}}

\hypertarget{http}{%
\subsubsection{HTTP}\label{http}}

\hypertarget{sldap}{%
\subsubsection{S/LDAP}\label{sldap}}

\hypertarget{racf}{%
\subsubsection{RACF}\label{racf}}

\hypertarget{xacml}{%
\subsubsection{XACML}\label{xacml}}

\hypertarget{technical}{%
\section{Technical}\label{technical}}

\hypertarget{repositories}{%
\subsection{Repositories}\label{repositories}}

\hypertarget{relational-database}{%
\subsubsection{Relational Database}\label{relational-database}}

\hypertarget{query-optimization}{%
\paragraph{Query optimization}\label{query-optimization}}

\hypertarget{replication-limitations}{%
\paragraph{Replication limitations}\label{replication-limitations}}

\hypertarget{directories}{%
\subsubsection{Directories}\label{directories}}

\hypertarget{historical-note---x.500}{%
\paragraph{Historical note - X.500}\label{historical-note---x.500}}

\hypertarget{slapd-and-its-descendants}{%
\paragraph{SLAPD and its
descendants}\label{slapd-and-its-descendants}}

\hypertarget{nosql-databases}{%
\subsubsection{NoSQL databases}\label{nosql-databases}}

\hypertarget{graph-databases}{%
\paragraph{Graph Databases}\label{graph-databases}}

\hypertarget{identity-provider-idp-trends}{%
\subsubsection{Identity Provider (IdP)
Trends}\label{identity-provider-idp-trends}}

\hypertarget{distributed-ledger-blockchain}{%
\paragraph{Distributed Ledger
(Blockchain)}\label{distributed-ledger-blockchain}}

\hypertarget{identity-provider-services}{%
\subsection{Identity Provider
Services}\label{identity-provider-services}}

\hypertarget{protocols}{%
\subsection{Protocols}\label{protocols}}

\hypertarget{kerberos}{%
\subsubsection{Kerberos}\label{kerberos}}

\hypertarget{lightweight-directory-access-protocol-ldap}{%
\subsubsection{Lightweight Directory Access Protocol
(LDAP)}\label{lightweight-directory-access-protocol-ldap}}

\hypertarget{scim}{%
\subsubsection{SCIM}\label{scim}}

\hypertarget{saml}{%
\subsubsection{SAML}\label{saml}}

\hypertarget{sp-initiated-vs-idp-initiated}{%
\paragraph{SP Initiated vs IDP
Initiated}\label{sp-initiated-vs-idp-initiated}}

\hypertarget{bindings}{%
\paragraph{Bindings}\label{bindings}}

\hypertarget{oidc}{%
\subsubsection{OIDC}\label{oidc}}

\hypertarget{authentications-flows}{%
\paragraph{Authentications Flows}\label{authentications-flows}}

\hypertarget{oauth}{%
\subsubsection{OAuth}\label{oauth}}

\hypertarget{ws-fed}{%
\subsubsection{WS-Fed}\label{ws-fed}}

\hypertarget{fido-u2f-and-uaf}{%
\subsubsection{FIDO U2F and UAF}\label{fido-u2f-and-uaf}}

\hypertarget{enterprise-control-of-cloud}{%
\subsection{Enterprise control of
``Cloud''}\label{enterprise-control-of-cloud}}

\hypertarget{public-cloud-vs-private-cloud}{%
\subsubsection{Public Cloud vs Private
Cloud}\label{public-cloud-vs-private-cloud}}

\hypertarget{local-connectors-and-gateways}{%
\subsubsection{Local Connectors and
Gateways}\label{local-connectors-and-gateways}}

\hypertarget{ipsec-vpn}{%
\subsubsection{IPSec VPN}\label{ipsec-vpn}}

\hypertarget{recommended-practices}{%
\section{Recommended Practices}\label{recommended-practices}}

\hypertarget{design-for-security}{%
\subsection{Design for security}\label{design-for-security}}

\hypertarget{governance-and-administration}{%
\section{Governance and
Administration}\label{governance-and-administration}}

\hypertarget{audit}{%
\subsection{Audit}\label{audit}}

\hypertarget{monitoring}{%
\subsection{Monitoring}\label{monitoring}}

\hypertarget{operational-considerations}{%
\chapter{Operational Considerations}\label{operational-considerations}}

\hypertarget{account-recovery}{%
\section{Account recovery}\label{account-recovery}}

\hypertarget{call-centers}{%
\section{Call centers}\label{call-centers}}

\hypertarget{engagement-of-user-for-their-own-security}{%
\section{Engagement of user for their own
security}\label{engagement-of-user-for-their-own-security}}

\hypertarget{security-events-and-operations}{%
\section{Security events and
operations}\label{security-events-and-operations}}

\hypertarget{project-management}{%
\chapter{Project Management}\label{project-management}}

\hypertarget{introduction-1}{%
\section{Introduction}\label{introduction-1}}

\hypertarget{importance-of-project-management}{%
\section{Importance of Project
Management}\label{importance-of-project-management}}

Abstract: Many Identity and Access Management (IAM) projects proceed
without a project manager. In these cases the IT group in charge of
identity management are left to deploy the required solution in the
absence of any overarching management. While this is sometimes seen as
the most expedient way to get a system installed or updated, it is
short-sighted and likely to cost the organisation more money in the
longer term. An IAM solution touches so many systems within an
organisation and is dependent on the current and planned condition of so
many applications that to deploy a solution without properly considering
the impact, managing the required resources and keeping management
advised of progress, will result in a substandard deployment.

Project management does have a cost, it is typically between 5-10\% of a
project's total expenditure but it represents the best return in
comparison to any other investment an organisation is likely to be
afforded.

\hypertarget{characteristics-of-a-project-manager}{%
\section{Characteristics of a Project
Manager}\label{characteristics-of-a-project-manager}}

Abstract: Too often, in the IT sector, a project manager is~ low-level
employee who is expected to bring a project in on time and within budget
with minimal assistance from upper management and minimal visibility
within the organization. In reality a project manager needs sufficient
resources to allow him or her to adequately monitor and manage their
project, and regular communications with a steering committee consisting
of representatives from upper management. There are two prime
characteristics that are essential to a project manager:

\begin{longtable}[]{@{}ll@{}}
\toprule
Predictability & Management doesn't like surprises. A project manager
should determine and report on a project's duration and related costs to
a defined degree of confidence.\tabularnewline
\midrule
\endhead
Flexibility & Gone are the days when a project manager slavishly
followed an approved Gantt chart to the detriment of anyone who wants a
change. IT projects will typically undergo several baseline changes
during execution to accommodate: scope changes, dependencies on other
projects and changes in resource availability.\tabularnewline
\bottomrule
\end{longtable}

Project managers require competence in the five components of project
management:

-~ ~ ~ ~ ~ Planning

-~ ~ ~ ~ ~ Organising

-~ ~ ~ ~ ~ Resourcing

-~ ~ ~ ~ ~ Directing

-~ ~ ~ ~ ~ Controlling

\hypertarget{pmi-framework}{%
\section{PMI Framework}\label{pmi-framework}}

Abstract: By definition a project must have a start and a finish.
Business-as-usual is never project work and does not require the skills
of a project manager. Before the start of a project there will be some
preparatory work to define the concept. Between the commencement and
completion there are discrete stages that define the project work. After
the project completion the deliverable will enter an operational status.

\hypertarget{concept}{%
\subsection{Concept}\label{concept}}

Abstract: Projects come out of a need. In the IAM world it might be a
need to reduce costs and improve security by better using identity
information for on-boarding and off-boarding staff, it might be
improving governance over identity information or it might be upgrading
existing IAM infrastructure. Typically it will fall to a project sponsor
to communicate the requirement and commence evaluating the cost and
duration of the required activity. The sponsor will typically fund this
stage and then engage a project manager to complete the planning stage.

\hypertarget{planning-stage}{%
\subsection{Planning Stage}\label{planning-stage}}

Abstract: Once approval to proceed has been received the project manager
will engage with the stakeholders to define the project scope. It is
usual for the size and complexity of the project to increase at this
point. For an IAM project that might have initially been to deploy an
identity manager for the assignment of email accounts and AD account
will expand to include provisioning into corporate applications and
possibly include additional functionality such as periodic attestation
reporting and re-certification. It is important that the appropriate
project stakeholders have been engaged by this point, to ensure
appropriate definition of the project scope.

\hypertarget{deployment-stage}{%
\subsection{Deployment Stage}\label{deployment-stage}}

Abstract: The project deployment will vary depending upon the project
management mechanism to be used.

\hypertarget{methodologies}{%
\subsection{Methodologies}\label{methodologies}}

\hypertarget{pmo-issues}{%
\section{PMO Issues}\label{pmo-issues}}

Abstract: In organizations with a Project Management Office an IAM
project must follow the corporate procedures. Typically a PMO will have
defined ``gates'' through which all projects must pass. For instance
there will typically be a project approval gate in which the appropriate
managers will review the project plan and indicate their approval. There
will usually be a budget review to approve the assignment of resources.
There might be an architecture review to approve the solution
architecture. A review of the governance outcomes should also occur. The
PMO should orchestrate this activity.

One or the benefits of a PMO is the visibility it gives to projects
within an organization. This is beneficial to the IAM team in that it
gives them insight into which projects are proceeding and provides the
opportunity to ensure any projects with an identity component are
properly identified and accommodated in the appropriate program of work.
For instance, if an authentication gateway is being installed, any
application undergoing development should be modified to use the gateway
rather than maintaining LDAP lookups. Without a PMO it is sometimes
difficult for the IAM team to impact projects. A PMO provides the
opportunity to educate project managers on identity issues and to insert
IAM requirements into IT projects within an organisation.

~

\hypertarget{iam-knowledge-sharing}{%
\chapter{IAM Knowledge Sharing}\label{iam-knowledge-sharing}}

\hypertarget{idpro}{%
\section{IDPro}\label{idpro}}

\hypertarget{gartner}{%
\section{Gartner}\label{gartner}}

\hypertarget{kuppingercole}{%
\section{KuppingerCole}\label{kuppingercole}}

\hypertarget{iiw}{%
\section{IIW}\label{iiw}}

\hypertarget{bibliography}{%
\section{Bibliography}\label{bibliography}}

\hypertarget{advanced-topics-parking-lot}{%
\chapter{Advanced Topics -- Parking
Lot}\label{advanced-topics-parking-lot}}

\hypertarget{digital-legacy---handling-deceased-persons-digital-id-advanced-topic}{%
\section{Digital Legacy - handling deceased persons' digital ID
(Advanced
Topic)}\label{digital-legacy---handling-deceased-persons-digital-id-advanced-topic}}

\hypertarget{self-sovereign-identity}{%
\section{Self-Sovereign Identity}\label{self-sovereign-identity}}

\hypertarget{blockchain-id}{%
\subsection{Blockchain ID}\label{blockchain-id}}
